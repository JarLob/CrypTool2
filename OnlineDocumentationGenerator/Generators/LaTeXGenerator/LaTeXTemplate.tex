\documentclass[10pt,a4paper]{scrreprt}
\usepackage[utf8]{inputenc}
\usepackage{amsmath}
\usepackage{tipa}
\usepackage{amsfonts}
\usepackage{german}
\usepackage{graphicx}
\usepackage{amssymb}
\usepackage{parskip}
\usepackage{german,longtable}
\setcounter{tocdepth}{4}
\setcounter{secnumdepth}{4}

\makeatletter \setlength\@fptop{0\p@}
\makeatother

\title{Dokumentation zu den Vorlagen (Templates) in CrypTool v2 (CT2)}
\author{Das CrypTool 2 Team}

\newcommand{\HRule}{\rule{\linewidth}{0.5mm}}

\begin{document}

\begin{titlepage}
\begin{center}
\hspace{0pt}\\[2.5cm]

\HRule \\[0.4cm]
{ \huge \bfseries Dokumentation zu den Templates in CrypTool v2 (CT2) }\\[0.4cm]
\HRule \\[1.5cm]

\begin{minipage}{0.4\textwidth}
\begin{flushleft} \large
\emph{Autoren:} \\
Das CrypTool 2 Team
\end{flushleft}
\end{minipage}
\begin{minipage}{0.4\textwidth}
\begin{flushright} \large
\emph{Version:} \\
$VERSION$
\end{flushright}
\end{minipage}
\vfill
{\large \today}
\end{center}
\newpage
\large
Dieses Dokument gibt eine �bersicht �ber die in {\bf CrypTool v2} enthaltenen Vorlagen.\\\\
Der erste Teil dieses Dokumentes besteht aus einer kompakten Auflistung des Titels und einer kurzen Zusammenfassung (Summary) der Vorlagen in derselben Reihenfolge, wie sie im Startcenter von CT2 gruppiert sind.\\\\
Ab Seite \pageref{part2} wird zus�tzlich zu jeder Vorlage der im Programm verwendete Balloontext, der eine l�ngere Beschreibung der Vorlage enth�lt, aufgef�hrt.
\newpage
\end{titlepage}

\tableofcontents
\newpage

$CONTENT$

\end{document}
