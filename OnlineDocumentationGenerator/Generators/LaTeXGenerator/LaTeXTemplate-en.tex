\documentclass[10pt,a4paper]{scrreprt}
\usepackage[utf8]{inputenc}
\usepackage{amsmath}
\usepackage{tipa}
\usepackage{amsfonts}
\usepackage[english]{babel}
\usepackage{graphicx}
\usepackage{amssymb}
\usepackage{parskip}
\usepackage{german,longtable}
\setcounter{tocdepth}{4}
\setcounter{secnumdepth}{4}

\makeatletter \setlength\@fptop{0\p@}
\makeatother

\title{Documentation of the Templates in CrypTool v2 (CT2)}
\author{The CrypTool 2 Team}

\newcommand{\HRule}{\rule{\linewidth}{0.5mm}}

\begin{document}

\begin{titlepage}
\begin{center}
\hspace{0pt}\\[2.5cm]

\HRule \\[0.4cm]
{ \huge \bfseries Documentation of the Templates in CrypTool v2 (CT2) }\\[0.4cm]
\HRule \\[1.5cm]

\begin{minipage}{0.4\textwidth}
\begin{flushleft} \large
\emph{Authors:} \\
The CrypTool 2 Team
\end{flushleft}
\end{minipage}
\begin{minipage}{0.4\textwidth}
\begin{flushright} \large
\emph{Version:} \\
$VERSION$
\end{flushright}
\end{minipage}
\vfill
{\large \today}
\end{center}
\newpage
\large
This document gives an overview over the templates that are included in {\bf CrypTool v2}.\\\\
The first part of this document (``Overview over the Templates'') compactly lists all templates. For each template, it lists the icon, the title and a short summary. The templates are presented in the same order as they appear in the Startcenter of CT2.\\\\
On page \pageref{part2} starts a list of the templates and the corresponding balloon text that is used in the program and which offers a more detailed description of the template.
\newpage
\end{titlepage}

\tableofcontents
\newpage

$CONTENT$

\end{document}
